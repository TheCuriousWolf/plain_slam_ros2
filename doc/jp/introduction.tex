\chapter{はじめに}

\section{背景知識}

ロボットの自律移動や自動車の自動運転を行うにあたっては,地図を構築し,その地図上で走行している位置を認識する技術,いわゆる{\bf 自己位置推定}(Localization)や{\bf Simultaneous Localization and Mapping}(SLAM)とよばれる技術が重要であるとされています\cite{Thrun:2005:PR:1121596}.
従来,これらの技術を用いる場合は,{\bf オドメトリ}(Odometry)と呼ばれる移動量推定を行う枠組みが用いられていました.
自己位置推定やSLAMで用いられている技術を端的に述べると,構築された地図(SLAMの場合はオンラインで構築している地図)と,センサの観測値を照合することで,地図上のどの位置に自分が存在しているかを認識する技術になります.
この「センサの観測値と地図の照合」を行うにあたり,オドメトリから予測された移動量を用いることで,どの程度移動したかを予測することが可能となり,照合を行う際の探索範囲を限定することができるようになります.
そのためオドメトリを用いると,自己位置推定およびSLAMの精度や頑健性を向上させることができます.

しかしオドメトリを用いるとなると,自己位置推定やSLAMで用いられる外界センサ(LiDARやカメラ)以外のセンサが必要になります\cite{BorensteinJRS1997}.
オドメトリシステムを構築する上で最も簡単な方法(システムを構築する手間が最もかからないという意味で)は,{\bf Inertial Measurement Unit}(IMU)を用いることだといえます.
IMUは,センサを原点とした加速度と角速度を計測できるセンサであり,これらの値を積分していくだけで移動量を計算することができます.
しかしIMUの計測値が含む誤差は大きく,単に積分して得られた位置や角度の精度は極めて低く,自己位置推定やSLAMには利用できないことがほとんどです.
そのため,IMUだけを用いてオドメトリシステムを構築することは不可能に近いといえます.

移動ロボットや自動運転の分野で最も広く使われているオドメトリシステムは,エンコーダ等を用いて車輪の回転量を計測し,その結果を積分することで移動量を計算する方法です.
これは{\bf ホイールオドメトリ}(Wheel Odometry)と呼ばれます.
ホイールオドメトリを用いれば,タイヤの空転等が発生しない限り,短距離であれば移動量を正確に計測することができます.
ただしホイールオドメトリを用るには,外界センサだけでなく,移動体のハードウェアにも大きな変更を加える必要がります.
そのためホイールオドメトリは,安易に利用できるシステムとは言い難いです.
またホイールオドメトリは,車輪の回転量を計測することが前提のため,基本的に車輪型の移動体にしか適用することができません.

ホイールオドメトリに頼らない移動量の推定方法として,{\bf ビジュアルオドメトリ}(Visual Odometry)が提案されました\cite{VisuailOdometry}.
ビジュアルオドメトリとは,画像から得られる特徴を追跡することで移動量を推定する方法です.
そのためビジュアルオドメトリは,車輪型以外の移動体にも適用することが可能です.
しかし一般に,ビジュアルオドメトリの精度はホイールオドメトリ程高くはないことが知られています.

同様にLiDARを用いて移動量推定を行う方法も様々研究されていましたが,本格的に{\bf LiDAR Odometry}(LO)という言葉が使われた始めた代表的な手法としてLiDAR Odometry and Mapping(LOAM)があります\cite{LOAM}.
LOでは,LiDARが計測する点群を逐次的に照合していくことで移動量の推定を行います.
LiDARの距離計測の精度は高いため,LOによる移動量推定の精度は高くなります.
そのためLOは,様々な用途で使われるようになりました.

しかしLOにも弱点がありました.
LiDAR,特に3D LiDARの計測周期は遅く(一般に$10 \sim 20~{\rm Hz}$程度),高速な動き,特に回転を含む移動量を推定することは困難でした.
この問題を解決する方法として提案されたのが,LiDARとIMUを融合して移動量推定を行う{\bf LiDAR--Inertial Odometry}(LIO)です.
IMUは高周期(一般に100~Hz以上)で加速度と角速度を計測することができるため,LiDARの計測周期の移動量を補間することができます.
この移動の補間を用いると,高速に移動するLiDARによって歪んでしまったLiDARの計測点群を補正することができるようになります.
またLIOでは,LOでは推定されていなかった状態量(速度やIMUの計測値のバイアス)の推定も行います.
そのため,IMUの計測値の積分も正確に行えるようになるため,移動量の推定をより高精度に行うことが可能になりました.
なおVisionとIMUを融合した{\bf Visual--Inertial Odometry}(VIO)は,LIOよりも少し早く提案されていました(例えば\cite{VIO}).









\section{LIOの性能と限界}

LIOのアルゴリズムの進化は目覚ましいものがありましたが,LiDARの性能自体もここ数年で大きく進化しました.
一昔前(著者が研究を始めたのが2011年)では,「LiDARは価格コストが高いため,カメラを用いた手法を提案する」というのが論文等では常套文句でした.
しかし今では,日本円で10万円程度で購入可能な3D LiDARも発売されています.
そして驚くことに,このような価格の3D LiDARでも,360度100~mに近いレンジを計測できるようになっており,LIOを用いて高精度な移動量推定を行うということはかなり一般的になってきました.
そしてLIOを用いるだけでも,小規模な環境であれば十分な精度の点群地図を構築することができるようになりました.
そのため,ドローンのような飛翔体にこのような小型のLiDARを搭載し,高精度な点群地図を生成することも容易に行われるようになってきました.

ただしLIOはあくまでオドメトリシステムであるため,移動量推定しか行いません.
そのため,どれだけ高精度に移動量が推定できたとしても,推定量に誤差(ドリフト)が含まれてしまうため,LIOだけを用いて大規模な環境の地図構築を行うことは依然として難しいです.
特に大規模でループ(一度通過した地点を再度通過すること)が含まれる環境や条件ですと,整合性の取れた地図が構築できなくなってしまいます(同じものが同じ地点に正しくマッピングされなくなります).
前述したSLAMでは,このようなループが含まれる場合であっても,整合性が取れた地図構築を行うことを目的としています.
すなわち,精度の高い地図を構築したい場合には,SLAMの利用は避けられません.

またLIOはあくまで移動量推定のシステムです.
応用上においては,移動量がわかるだけでは嬉しさがあることは少ないといえ,SLAMで構築した地図上で,どの位置にいるかを知れることの方が恩恵が多いといえます.
例えば工場等でAGVやフォークリフトの位置を管理したい場合などには,LIOの利用だけでは不十分であり,自己位置推定の利用が求められます.
そのため,単に高精度のLIOが利用可能になったというだけでは新たな応用システムを提案することは難しく,LIOに含まれるアルゴリズムを正しく理解し,それをSLAMや自己位置推定にも応用していくことが重要になります.





\section{既存手法と本書の立ち位置}

LIOやSLAMを行うオープンソースはすでに多数存在しています.
例えばLIOの有名なオープンソースとしてはLIO-SAM~\cite{liosam2020shan}やFAST-LIO~\cite{FAST-LIO2}が挙げられます.
これらの手法の性能は極めて高く,これらをダウンロードして使用するだけでも,十分な移動量推定を行うことができるといえます.
またLIO-SAMにはSLAMの機能も含まれているため,地図構築を行うこともできます.
またLiDAR SLAMの有名なオープンソースとしては,GLIM~\cite{KoideRAS2024}が挙げられます.
これらの性能も極めて高く,様々な環境で極めて精度の高い点群地図を構築することができます.

しかし多くのソースコードは,機能を多く含むため,どうしても規模が大きくなってしまいます.
そのため,初めてSLAMを学ぼうとする人がこれらを見ても,どこから何を追えば良いかの判断が難しく,結局ダウンロードして使うだけになってしまうことが多いと思います.
本書,および対応するソースコードは,ソフトウェアの構成をとにかくシンプルに実装することに重きをおいています.
開発したソースコードは,LIO,SLAM,自己位置推定の機能を有しており,その主な処理は空行を除いて2000行未満のC++で完結しています.
このC++の中に,スキャンマッチング,LiDARとIMUの融合(ルーズカップリングとタイトカップリング),ループ検知,ポーズグラフの最適化といった必要な処理をすべて実装しています(用語の詳細は後ほど解説します).
また依存ライブラリも極力少なくし,ほぼフルスクラッチでLIOやSLAMを実装できるようになるようにしています.
なお,主な依存ライブラリは{\it Sophus}({\it Eigen}ベース)と{\it nanoflann}のみになります(他はパラメーター設定のために{\it YAML}を用いています).
これらはそれぞれ線形・リー代数を扱うライブラリと,最近某探索を行うライブラリとなっており,LIOやSLAMの根幹となる部分はすべてフルスクラッチで実装されているため,初学者でもどのように最適化などが実装されているか理解しやすい構成になっていると思います.


